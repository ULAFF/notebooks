\documentclass{article}

\input{flatex}   % include all the FLaTeX commands

\begin{document}

% The below figure in the \LaTeX document will be filled in by 
% defining a number of commands.  

\resetsteps      % Reset all the commands to create a blank worksheet  

% I like defining these kinds of commands to make most of the source
% as generic as possible

\newcommand{\ArgOne}{L}
\newcommand{\ArgTwo}{B}

% Fill in the operation to be performed

\renewcommand{\operation}{ \ArgTwo \becomes \ArgOne^{-1} \ArgTwo }

% Step 1a: Fill in the precondition

\newcommand{\OrigTwo}{ \hat{\ArgTwo} }  % set \OrgTwo = \hat{B} 

\renewcommand{\precondition}{ 
\ArgTwo = \hat{\OrigTwo}                % B = \hat{B}
}

% Step 1b: Fill in the postcondition

\renewcommand{\postcondition}{ 
\ArgTwo = \ArgOne^{-1} \hat{\OrigTwo}   % B = L^{-1} \hat{B}
}

% Step 2: Fill in the loop-invariant

\newcommand{\PartArgTwo}{               % Partition B into top and bottom
\FlaTwoByOne 
  { \ArgTwo_{T} }
  { \ArgTwo_{B} }
}

\renewcommand{\invariant}{              % Give the invariant
\PartArgTwo
    =
\FlaTwoByOne 
  { L_{TL}^{-1} \hat{B}_{T} }
  { \hat{B}_{B} - L_{BL} L_{TL}^{-1} \hat{B}_T }
}

% Step 3: Fill in the loop-guard

\renewcommand{\guard}{
n( L_{TL} ) \neq n( L )
}

% Step 4: Fill in the initialization

\newcommand{\PartArgOne}{               % Partition L into quadrants
\FlaTwoByTwo
  { \ArgOne_{TL} }{   0     }
  { \ArgOne_{BL} }{ \ArgOne_{BR} }
}

\newcommand{\PartOrigTwo}{              % Partition \hat{B} into top and bottom
\FlaTwoByOne 
  { \OrigTwo_{T} }
  { \OrigTwo_{B} }
}

% Insert the partitionings into the Initialization

\renewcommand{\partitionings}{
$ \ArgOne \rightarrow \PartArgOne $,    % Partition L
$ \ArgTwo \rightarrow \PartArgTwo $,    % Partition B
\\                       % move to next line
and
$ \OrigTwo \rightarrow \PartOrigTwo $  % Partition B
}

% Insert the sizes of the partitionings into the initialization

\renewcommand{\partitionsizes}{
$ \ArgOne_{TL} $ is $ 0 \times 0 $, and 
$ \ArgTwo_{T} $ and $ \OrigTwo_{T} $
have $ 0 $ rows
}

% Step 5a: Repartition the partitioned matrices

\newcommand{\RepartArgOne}{
\FlaThreeByThreeBR
  { L_{00} }  {     0        }{ 0 }
  { l_{10}^T }{ \lambda_{11} }{ 0 }
  { L_{20}   }{ l_{21}       }{ L_{22} }
}

\newcommand{\RepartArgTwo}{
\FlaThreeByOneB
  { B_{0} }  
  { b_{1}^T }
  { B_{2}   }
}

\newcommand{\RepartOrigTwo}{
\FlaThreeByOneB
  { \hat{B}_{0} }  
  { \hat{b}_{1}^T }
  { \hat{B}_{2}   }
}

\renewcommand{\repartitionings}{
 $  \PartArgOne \rightarrow \RepartArgOne $,
 $  \PartArgTwo \rightarrow \RepartArgTwo $, \\
 and
 $  \PartOrigTwo \rightarrow \RepartOrigTwo $
}

\renewcommand{\repartitionsizes}{
$ \lambda_{11} $ is a scalar and $ b_1^T $ and $ \hat{b}_1^T $
are rows
}

% Step 5b: ``Move the double lines''

\newcommand{\MoveArgOne}{
\FlaThreeByThreeTL
  { L_{00} }  {     0        }{ 0 }
  { l_{10}^T }{ \lambda_{11} }{ 0 }
  { L_{20}   }{ l_{21}       }{ L_{22} }
}

\newcommand{\MoveArgTwo}{
\FlaThreeByOneT
  { B_{0} }  
  { b_{1}^T }
  { B_{2}   }
}

\newcommand{\MoveOrigTwo}{
\FlaThreeByOneT
  { \hat{B}_{0} }  
  { \hat{b}_{1}^T }
  { \hat{B}_{2}   }
}

\renewcommand{\moveboundaries}{
 $  \PartArgOne \leftarrow \MoveArgOne $,
 $  \PartArgTwo \leftarrow \MoveArgTwo $, \\
 and
 $  \PartOrigTwo \leftarrow \MoveOrigTwo $
}

% Step 6: State after the repartitioning

\renewcommand{\beforeupdate}{
 \RepartArgTwo =
 \FlaThreeByOneB
   { L_{00}^{-1} \hat{B}_0 }
   { \hat{b}_1^T - l_{10}^T L_{00}^{-1} \hat{B}_0 }
   { \hat{B}_2   - L_{20}   L_{00}^{-1} \hat{B}_0 }
}

% Step 7: State after moving the double lines

\renewcommand{\afterupdate}{
 \MoveArgTwo =
 \FlaThreeByOneT
   { L_{00}^{-1} \hat{B}_0 }
   { \lambda_{11}^{-1}( \hat{b}_1^T - l_{10}^T L_{00}^{-1} \hat{B}_0 ) }
   { \hat{B}_2   - L_{20}   L_{00}^{-1} \hat{B}_0 
                 - l_{21} \lambda_{11}^{-1}( \hat{b}_1^T - l_{10}^T L_{00}^{-1} \hat{B}_0 ) }
}

% Step 8: Fill in the update
\renewcommand{\update}{
$
  \begin{array}{l}
    b_{1}^T \becomes b_{1}^T / \lambda_{11} \\
    B_{2}   \becomes B_{2} - l_{21} b_{1}^T
  \end{array}
$
}

% Typically, in a document, an algorithm will appear in a separate figure.
% For this we use the \LaTeX figure environment.  The optional arguments
% [tbp] indicate that the figure should be placed at the first of
% the top of a page, the bottom of a page, or on a page by itself, in
% that order.

\begin{figure}[tbp]

\begin{center}     % center the algorithm on the page
\footnotesize      % make the font smaller so things fit on a page

\worksheet         % this command generates the worksheet using the
                   % commands that were defined between the \resetsteps
                   % and the \worksheet command
\end{center}

% The following command creates a caption for the figure.

\caption{ Worksheet for deriving unblocked algorithm for
$ B \becomes L^{-1} B $ (Variant 2).}


% The following command creates a label so that the figure can
% be referenced by its number by inserting 
%  \ref{Trsm_llnn:annotated_unb_var2} in the \LaTeX source

\label{Trsm_llnn:annotated_unb_var2}

\end{figure}

% Next, we create a figure with just the algorithms in it.
% To do so, we must redefine a few commands.  Commands
% related to annotations are simply not printed.  However,
% some other commands reference \hat{B}, which is only needed
% for that annotations as part of the argument of correctness.

% Insert the partitionings into the Initialization

% Step 4: Fill in the initialization

\renewcommand{\partitionings}{
$ \ArgOne \rightarrow \PartArgOne $, and   % Partition L 
$ \ArgTwo \rightarrow \PartArgTwo $        % Partition B
}

% Insert the sizes of the partitionings into the initialization

\renewcommand{\partitionsizes}{
$ \ArgOne_{TL} $ is $ 0 \times 0 $ and 
$ \ArgTwo_{T} $ has $ 0 $ rows
}

% Step 5a: Repartition the partitioned matrices

\renewcommand{\repartitionings}{
 $  \PartArgOne \rightarrow \RepartArgOne $ and
 $  \PartArgTwo \rightarrow \RepartArgTwo $
}

\renewcommand{\repartitionsizes}{
$ \lambda_{11} $ is a scalar and $ b_1^T $ is a row
}

% Step 5b: ``Move the double lines''

\renewcommand{\moveboundaries}{
 $  \PartArgOne \leftarrow \MoveArgOne $,
 $  \PartArgTwo \leftarrow \MoveArgTwo $, \\
 and
 $  \PartOrigTwo \leftarrow \MoveOrigTwo $
}

\begin{figure}[tbp]

\begin{center}     % center the algorithm on the page
\footnotesize      % make the font smaller so things fit on a page

\FlaAlgorithm      % this command generates the algorithm using the
                   % commands that were defined
\end{center}

% The following command creates a caption for the figure.

\caption{ Algorithm for deriving unblocked algorithm for
$ B \becomes L^{-1} B $ (Variant 2).}


% The following command creates a label so that the figure can
% be referenced by its number by inserting 
%  \ref{Trsm_llnn:alg_unb_var2} in the \LaTeX source

\label{Trsm_llnn:alg_unb_var2}

\end{figure}

\end{document}